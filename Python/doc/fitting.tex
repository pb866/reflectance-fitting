\documentclass[english]{scrartcl}
\usepackage[T1]{fontenc}
\usepackage[latin9]{inputenc}
\usepackage{babel}
\usepackage{graphicx}
\usepackage{listings}
\usepackage{amsmath}
\usepackage{url}
% 
\renewcommand{\tt}{\normalfont \ttfamily}

\begin{document}
\title{Fitting ALS Reflectance Data using Python}
\author{R. Steven Turley}
\date{April, 2018}

\maketitle
\tableofcontents{}

\section{Introduction}

This report explains how to use Python to fit reflectance data
taken in Beamline 6.3.2 at the Advanced Light Source at Lawrence
Berkeley National Lab. It is based on some classes I wrote to
make the processes somewhat easier.

\subsection{Python}
If you don't already know at least a little Python, I would suggest
reading a python tutorial before reading this. I'll explain
things a little bit along the way, but the purpose of this document
is more about fitting the data than about learning Python.

\subsection{Fitting Steps}
There are a couple of steps to the fitting process:
\begin{itemize}
\item reading in raw data
\item combining raw data files to compute $\theta$/$2\theta$ spectra
\item fitting the $\theta$/$2\theta$ spectra to theoretical curves
\item combining the fits to derive average data values
\item refitting the data with average data values.
\end{itemize}

\subsection{Refl Module}
The library routines and classes you'll need to do the fitting
are in the file \texttt{refl.py}. Before using these routines and
classes, you'll need to import with the entire module
\begin{lstlisting}
import refl
\end{lstlisting}
or the specific items you need from the module.
\begin{lstlisting}
from refl import Spectrum, Index, Parratt
\end{lstlisting}
The \texttt{refl} module contains the following items.
\begin{description}
\item [matR] a function to calculate reflectance and transmittance
	from a multilayer mirror using matrices. It is slower than
	Parratt if you only want the reflectance.
\item [Parratt] a function to calculated reflectance from a
	multilayer mirror.
\item [fracs] a function to calculate the fraction of s-polarized
	light in the synchrotron beam at the ALS. This is mostly used
	internally.
\item [Index] a class returning the complex index of refraction
	for a material.
\item [Run] a class with the raw data from a run.
\item [Runs] a cache of all of the run data form an ALS
	trip. Runs[i] is the Run numbered i.
\item [Reflectance] a class with computed reflectance from a set
	of runs.
\item [Log] a class with information from the log file saved
	with a set of runs.
\end{description}

\section{Reading Raw Data Files}
Raw data files from the ALS consists of a header line with data
information followed by four columns of numbers:
\begin{description}
\item [var] the data being varied during the measurement. For
	$\theta$/$2\theta$ measurements (our most common ones), this
	is the measurement angle. For $I_0$ measurements, this is the
	wavelength. For dark current measurements, this is usually
	the time.
\item [diode] the signal from the diode detector
\item [m3] the signal from the m3 mirror (usually ignored)
\item [beam] the beam current during the run (also usually ignored)
\end{description}
The Run objects are returned by the Runs class which reads the files
when needed and caches their contents for subsequent calls.

\section{Creating Reflectance Spectra}

\section{Fitting the Data}

\section{Finding Average Values}

\section{Refitting Data}

\end{document}