\documentclass{scrartcl}
\usepackage{amsmath}
\usepackage{graphicx}
\usepackage{url}
\title{Cubic Splines with Irregularly Spaced Points}
\subtitle{Version 0.1}
\author{R. Steven Turley}
\date{August 17, 2018}
\begin{document}
\maketitle
\tableofcontents

\section{Introduction}
This is the mathematics and some implementation details
behind a derivation of 1d cubic spline interpolation with
regularly and irregularly spaced points.

I will use the article on splines for a regularly-spaced
grid in MathWorld\cite{mathworld} as a basis for my
derivations and generalizations.

Splines are piece-wise cubic polynomials which are continuous
and have continuous first and second derivatives. In each interval
it takes four coefficients to define a cubic polynomial. If there
are $n+1$ points, there are $n$ intervals requiring $4n$ coefficients
for the splines. Let the knots on the spline (the data points that
match exactly) be $(x_i,y_i)$. Let $Y_i(x)$ be the cubic polynomial for
the interval $i$ where $x_i\leq x\leq x_{i+1}$. Then the $4n-4$ conditions
for matching the points and having continuous first and second derivatives
are for $2\leq n \leq n$
\begin{align}
Y_{i-1}(x_i) &= y_i \label{eq:cbegin} \\
Y_i(x_i) &= y_i\\
Y'_{i-1}(x_i) &= Y'_i(x_i)\\
Y''_{i-1}(x_i) &= Y''_i(x_i)\;. \label{eq:c2}
\end{align}
In addition to these equations, the spline also needs to match
at the two endpoints.
\begin{align}
Y_1{x_1} &= y_1\\
Y_n(x_{n+1}) &= y_{n+1}
\end{align}
This gives a total of $4n-2$ equations and $4n$ unknowns. There
are several ways to choose the last two conditions. I will use the
specification that the second derivative be zero at the two endpoints.
\begin{align}
Y''_1{x_1} &= 0\\
Y''_{n+1}(x_{n+1}) &= 0
\end{align}

\section{Regularly-Spaced Points}

The spline equations can be solved with a particularly elegant
form for the case of equally spaced knots. It is useful to put the
origin of each cubic at the beginning of the interval and transform
to a variable $t$ which goes from 0 to 1 in each interval $i$.
\begin{align}
x &= x_i + \alpha t \qquad\mbox{for}\quad x_i\leq x\leq x_{i+1}\label{eq:xt}\\
\alpha &= x_{i+1}-x_i \quad\forall i \label{eq:alpha}
\end{align}
If the
intervals are of equal length, the conditions of continuity
of a derivative with respect to $x$ is the same as a derivative
with respect to $t$. Equations~\ref{eq:cbegin} through \ref{eq:c2}
are then
\begin{align}
Y_{i-1}(1) &= y_i\\
Y_i(0) &= y_i\\
Y'_{i-1}(1) &= Y'_i(0)\\
Y''_{i-1}(1) &= Y''_i(0).
\end{align}
Let the four coefficients of the cubic for interval $i$ be given
by
\begin{equation}
Y_i(t) = a_i + b_i t + c_i t^2 + d_i t^3 .
\end{equation}
Then these coefficients can be solved for in terms of the
values $y_i$ and the derivatives $D_i = Y'_i(0)$.
\begin{align}
Y_i(0)&=y_i = a_i\\
Y_i(1)&=y_{i+1} = a_i + b_i + c_i + d_i\\
Y'_i(0) &= D_i = b_i\\
Y'_i(1) &= D_{i+1} = b_i + 2 c_i + 3 d_i
\end{align}
These equations can be solved for the cubic coefficients in
terms of $y_i$ and $D_i$.
\begin{align}
a_i &= y_i \label{eq:ai}\\
b_i &= D_i\\
c_i &= 3(y_{i+1}-y_i)-2D_i-D_{i+1}\\
d_i &= 2(y_i-y_{i+1})+D_i +D_{i+1} \label{eq:di}
\end{align}
Weisstein shows that these equations can be rewritten as
the matrix equation
\begin{equation}
\left(\begin{array}{ccccccc}
2&1\\
1&4&1\\
&1&4&1\\
&&1&4&1\\
\vdots&\ddots&\ddots&\ddots&\ddots&\ddots&\ddots\\
&&&&1&4&1\\
&&&&&1&2
\end{array}\right)
\left(\begin{array}{c}
D_1\\D_2\\D_3\\D_4\\ \vdots\\D_n\\D_{n+1}
\end{array}\right) =
\left(\begin{array}{c}
3(y_2-y_1)\\
3(y_3-y_1)\\
3(y_4-y_2)\\
\vdots\\
3(y_n-y_{n-2})\\
3(y_{n+1}-y_{n-1})\\
3(y_{n+1}-y_n)
\end{array}\right).\label{eq:eqtd}
\end{equation}
Equation~\ref{eq:eqtd} can be solved with an efficient symmetric
tridiagonal solver in Julia for the unknown values $D_i$. Once those
are known, Equations~\ref{eq:ai} through \ref{eq:di} can be used
to solve for $a_i$, $b_i$, $c_i$, and $d_i$.

\section{Irregularly-Spaced Points}
If the points $x_i$ are not regularly spaced, $\alpha$ in
Equations~\ref{eq:xt} and \ref{eq:alpha} needs to be replaced
with $\alpha_i$ which will vary in each interval.
\begin{align}
x &= x_i + \alpha_i t \qquad\mbox{for}\quad x_i\leq x\leq x_{i+1}\\
\alpha_i &= x_{i+1}-x_i \quad\forall i
\end{align}
In addition,
the first and second derivative equations for derivatives with
respect to $x$ are no longer the same as the conditions on the
derivatives with respect to $t$.
\begin{align}
Y'_i(x) &= Y'_i(t)\frac{dt}{dx}\\
&= \frac{Y'_i(t)}{\alpha_i}
\end{align}

\begin{thebibliography}{9}

\bibitem{mathworld}
Weisstein, Eric W. "Cubic Spline." From
\textit{MathWorld}--A Wolfram Web Resource.
\url{http://mathworld.wolfram.com/CubicSpline.html} (accessed
8/17/2018). Citing: Bartels, R. H.; Beatty, J. C.; and Barsky, B. A.
"Hermite and Cubic Spline Interpolation," Ch. 3 in
\textit{An Introduction to Splines for Use in Computer Graphics
and Geometric Modelling}. San Francisco, CA: Morgan Kaufmann,
pp. 9--17, 1998.

\end{thebibliography}

\end{document}