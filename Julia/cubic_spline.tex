\documentclass{scrartcl}
\usepackage{amsmath}
\usepackage{graphicx}
\usepackage{url}
\title{Cubic Splines with Irregularly Spaced Points}
\subtitle{Version 0.1}
\author{R. Steven Turley}
\date{August 17, 2018}
\begin{document}
\maketitle
\tableofcontents

\section{Introduction}
This is the mathematics and some implementation details
behind a derivation of 1d cubic spline interpolation with
regularly and irregularly spaced points.

I will use the article on splines for a regularly-spaced
grid in MathWorld\cite{mathworld} as a basis for my
derivations and generalizations.

Splines are piece-wise cubic polynomials which are continuous
and have continuous first and second derivatives. In each interval
it takes four coefficients to define a cubic polynomial. If there
are $n+1$ points, there are $n$ intervals requiring $4n$ coefficients
for the splines. Let the knots on the spline (the data points that
match exactly) be $(x_i,y_i)$. Let $Y_i(x)$ be the cubic polynomial for
the interval $i$ where $x_i\leq x\leq x_{i+1}$. Then the $4n-4$ conditions
for matching the points and having continuous first and second derivatives
are for $2\leq n \leq n$
\begin{align}
Y_{i-1}(x_i) &= y_i \label{eq:cbegin} \\
Y_i(x_i) &= y_i\\
Y'_{i-1}(x_i) &= Y'_i(x_i)\\
Y''_{i-1}(x_i) &= Y''_i(x_i)\;. \label{eq:c2}
\end{align}
In addition to these equations, the spline also needs to match
at the two endpoints.
\begin{align}
Y_1(x_1) &= y_1\\
Y_n(x_{n+1}) &= y_{n+1}
\end{align}
This gives a total of $4n-2$ equations and $4n$ unknowns. There
are several ways to choose the last two conditions. I will use the
specification that the second derivative be zero at the two endpoints.
\begin{align}
Y''_1(x_1) &= 0 \label{eq:bslope}\\
Y''_{n+1}(x_{n+1}) &= 0 \label{eq:eslope}
\end{align}

\section{Regularly-Spaced Points}

The spline equations can be solved with a particularly elegant
form for the case of equally spaced knots. It is useful to put the
origin of each cubic at the beginning of the interval and transform
to a variable $t$ which goes from 0 to 1 in each interval $i$.
\begin{align}
x &= x_i + \alpha t \qquad\mbox{for}\quad x_i\leq x\leq x_{i+1}\label{eq:xt}\\
\alpha &= x_{i+1}-x_i \quad\forall i \label{eq:alpha}
\end{align}
If the
intervals are of equal length, the conditions of continuity
of a derivative with respect to $x$ is the same as a derivative
with respect to $t$. Equations~\ref{eq:cbegin} through \ref{eq:c2}
are then
\begin{align}
Y_{i-1}(1) &= y_i\\
Y_i(0) &= y_i\\
Y'_{i-1}(1) &= Y'_i(0)\\
Y''_{i-1}(1) &= Y''_i(0).
\end{align}
Let the four coefficients of the cubic for interval $i$ be given
by
\begin{equation}
Y_i(t) = a_i + b_i t + c_i t^2 + d_i t^3 .
\end{equation}
Then these coefficients can be solved for in terms of the
values $y_i$ and the derivatives $D_i = Y'_i(0)$.
\begin{align}
Y_i(0)&=y_i = a_i \label{eq:Yi0}\\
Y_i(1)&=y_{i+1} = a_i + b_i + c_i + d_i\\
Y'_i(0) &= D_i = b_i\\
Y'_i(1) &= D_{i+1} = b_i + 2 c_i + 3 d_i \label{eq:Ypi1}
\end{align}
These equations can be solved for the cubic coefficients in
terms of $y_i$ and $D_i$.
\begin{align}
a_i &= y_i \label{eq:ai}\\
b_i &= D_i\\
c_i &= 3(y_{i+1}-y_i)-2D_i-D_{i+1} \label{eq:ci}\\
d_i &= 2(y_i-y_{i+1})+D_i +D_{i+1} \label{eq:di}
\end{align}
Weisstein shows that these equations can be rewritten as
the matrix equation
\begin{equation}
\left(\begin{array}{ccccccc}
2&1\\
1&4&1\\
&1&4&1\\
&&1&4&1\\
\vdots&\ddots&\ddots&\ddots&\ddots&\ddots&\ddots\\
&&&&1&4&1\\
&&&&&1&2
\end{array}\right)
\left(\begin{array}{c}
D_1\\D_2\\D_3\\D_4\\ \vdots\\D_n\\D_{n+1}
\end{array}\right) =
\left(\begin{array}{c}
3(y_2-y_1)\\
3(y_3-y_1)\\
3(y_4-y_2)\\
\vdots\\
3(y_n-y_{n-2})\\
3(y_{n+1}-y_{n-1})\\
3(y_{n+1}-y_n)
\end{array}\right).\label{eq:eqtd}
\end{equation}
My derivation of this is in Appendix~\ref{sec:reg-deriv}.
Equation~\ref{eq:eqtd} can be solved with an efficient symmetric
tridiagonal solver in Julia for the unknown values $D_i$. Once those
are known, Equations~\ref{eq:ai} through \ref{eq:di} can be used
to solve for $a_i$, $b_i$, $c_i$, and $d_i$.

\section{Irregularly-Spaced Points}
If the points $x_i$ are not regularly spaced, $\alpha$ in
Equations~\ref{eq:xt} and \ref{eq:alpha} needs to be replaced
with $\alpha_i$ which will vary in each interval.
\begin{align}
x &= x_i + \alpha_i t \qquad\mbox{for}\quad x_i\leq x\leq x_{i+1}\\
\alpha_i &= x_{i+1}-x_i \quad\forall i
\end{align}
In addition,
the first and second derivative equations for derivatives with
respect to $x$ are no longer the same as the conditions on the
derivatives with respect to $t$.
\begin{align}
Y'_i(x) &= Y'_i(t)\frac{dt}{dx}\\
&= \frac{Y'_i(t)}{\alpha_i}\\
Y''_i(x) &= Y''_i(t)\left(\frac{dt}{dx}\right)^2 +
	Y'_i(t)\frac{d^2t}{dx^2}\\
	&= \frac{Y''_i(t)}{\alpha_i^2}
\end{align}
With this change, Equations~\ref{eq:Yi0} through \ref{eq:Ypi1}
become
\begin{align}
Y_i(0)&=y_i = a_i \label{eq:Yi0a}\\
Y_i(1)&=y_{i+1} = a_i + b_i + c_i + d_i\\
Y'_i(0) &= D_i = b_i\\
\frac{Y'_i(1)}{\alpha_i} &= \frac{D_{i+1}}{\alpha_{i+1}}
 = \frac{b_i + 2 c_i + 3 d_i}{\alpha_i}. \label{eq:Ypi1a}
\end{align}
Note that Equation~\ref{eq:Ypi1a} predicts possible
wild behavior in the derivative if
a region with a large $\alpha_i$ next to a region
with a small $\alpha_i$. In this case, it might make sense to
use a linear interpolation between those points, even though
that will result in discontinuous slopes and second derivatives
at the knots. Appendix~\ref{sec:irreg-deriv} derives the following
matrix equation as a solution for $D_i$ in terms of
$y_i$ and $\alpha_i$.
\begin{align}
\left(\begin{array}{ccccc}
2&1\\
  \alpha_1^{-2}&2(\alpha_1^{-2}+\alpha_2^{-2})&\alpha_2^{-2}\\
 &\alpha_2^{-2}&2(\alpha_2^{-2}+\alpha_3^{-2})&\alpha_3^{-2}\\
\vdots&\ddots&\ddots&\ddots&\ddots\\
&&\alpha_{n-1}^{-2}&2(\alpha_{n-1}^{-2}+\alpha_n^{-2})&\alpha_n^{-2}\\
&&&1&2
\end{array}\right)
\left(\begin{array}{c}
D_1\\D_2\\D_3\\ \vdots\\D_n\\D_{n+1}
\end{array}\right) =\\
\left(\begin{array}{c}
3(y_2-y_1)\\
3[y_3\alpha_2^{-2}
	+y_2(\alpha_1^{-2}
	-\alpha_2^{-2})
	-y_1\alpha_1^{-2}]\\
3[y_4\alpha_3^{-2}
	+y_3(\alpha_2^{-2}
	-\alpha_3^{-2})
	-y_2\alpha_2^{-2}]\\
\vdots\\
3[y_{n+1}\alpha_n^{-2}+y_n(\alpha_{n-1}^{-2}
	-\alpha_n^{-2})-y_{n-1}\alpha_{n-1}^{-2}]\\
3(y_{n+1}-y_n)
\end{array}\right).\label{eq:eqtdi}
\end{align}
% just to be safe for later

\section{Cubic Spline Results}
My implementation of these results(which isn't necessarily bug free)
shows excessive oscillation of the splines for irregularly spaced
points. Doing the same calculation in Matlab gives better results.
I could have an unstable algorithm, an error in the implementation,
or an error in the derivation. In any case, piece-wise hermite polynomial
may be a better way to go. The following references are in the Matlab
documentation of the pchip function\cite{Fritsch,Kahaner}.

\section{Piece-wise Hermit Polynomial Interpolation}
\subsection{Derivatives}
The formulas in the previous section require the computation of
numerical derivatives at each knot. With equally-spaced knots, this
is probably best done using the straightforward "three point formula."
\begin{equation}
f'(b) \approx \frac{f(c)-f(a)}{2h},
\end{equation}
where $a$ and $c$ are the points to the left and right of $b$
respectively. If the points are not equally spaced, a somewhat less
accurate formula can be found from the average of the forward
and backward difference formulas.
\begin{align}
f'(b) &\approx \frac{f'_b(b)+f'_f(b)}{2}\\
&= \frac{f(b)-f(a)}{2(b-a)}+\frac{f(c)-f(b)}{2(c-b)}\\
&= \frac{f(c)}{2(c-b)}+f(b)\left(\frac{1}{2(b-a)}-\frac{1}{2(c-b)}\right)
	-\frac{f(a)}{2(b-a)}\\
&= \frac{f(c)}{2(c-b)}+f(b)\frac{a+c-2b}{2(b-a)(c-b)}
	-\frac{f(a)}{2(b-a)}
\end{align}

\appendix
\section{Solution for Regularly-Spaced Points}\label{sec:reg-deriv}
The goal is to solve Equations~\ref{eq:Yi0} through \ref{eq:Ypi1}
by eliminating the cubic coefficients and only having equations
in terms of the $y_i$ and $D_i$ variables. We first need to
add two more equations.
\begin{align}
Y''_i(0) &= 2c_i\label{eq:ypp0}\\
Y''_i(1) &= Y''_{i+1}(0) = 2c_i+6d_i\label{eq:yppn}\\
c_{i+1} &= c_i + 3d_i. \label{eq:cd}
\end{align}
Substituting in the values for $c_i$ from Equation~\ref{eq:ci} and
$d_i$ from Equation~\ref{eq:di}, Equation~\ref{eq:cd} becomes
\begin{multline}
3(y_{i+2}-y_{i+1})-2D_{i+1}-D_{i+2} = 3(y_{i+1}-y_i)-2D_i-D_{i+1}\\
	+3[2(y_i-y_{i+1})+D_i+D_{i+1}].
\end{multline}
Grouping the $y$ variables on one side of the equation and the $D$
variables on the other,
\begin{equation}
3(y_{i+2}-y_i) = D_{i+2} +4D{i+1} +D_i.
\end{equation}
This accounts for the middle rows of Equation~\ref{eq:eqtd}. The
top and bottom rows come from the initial and final conditions
in Equations~\ref{eq:bslope} and \ref{eq:eslope}. Combining
Equations~\ref{eq:bslope}, \ref{eq:ypp0}, and \ref{eq:ci},
\begin{align}
2c_1 & = 0\\
3(y_2-y_1) &= 2D_1 + D_2,
\end{align}
which is the first row of Equation~\ref{eq:eqtd}. Combining
Equations~\ref{eq:eslope}, \ref{eq:yppn}, \ref{eq:ci}, and
\ref{eq:di},
\begin{align}
2c_n+6d_n &= 0\\
3(y_{n+1}-y_n)-2D_n-D_{n+1}+3[2(y_n-y_{n+1})+D_n+D_{n+1}] &= 0\\
3(y_n-y_{n+1}) &= D_n+2D_{n+1},
\end{align}
which is the bottom row of Equation~\ref{eq:eqtd}.

\section{Solution for Irregularly-Spaced Points}\label{sec:irreg-deriv}
The goal is to solve Equations~\ref{eq:Yi0} through \ref{eq:Ypi1}
by eliminating the cubic coefficients and only having equations
in terms of the $y_i$, $D_i$, and $\alpha_i$ variables.
We first need to add two more equations.
\begin{align}
Y''_i(0) &= 2c_i\label{eq:ypp0i}\\
\frac{Y''_i(1)}{\alpha_i^2} &= \frac{Y''_{i+1}(0)}{\alpha_{i+1}^2}
 = \frac{2c_i+6d_i}{\alpha_i^2}\label{eq:yppni}\\
\frac{c_{i+1}}{\alpha_{i+1}^2} &=
	\frac{c_i + 3d_i}{\alpha_i^2}. \label{eq:cdi}
\end{align}
Substituting in the values for $c_i$ from Equation~\ref{eq:ci} and
$d_i$ from Equation~\ref{eq:di}, Equation~\ref{eq:cdi} becomes
\begin{multline}
\frac{3(y_{i+2}-y_{i+1})-2D_{i+1}-D_{i+2}}{\alpha_{i+1}^2} = \\
	\frac{3(y_{i+1}-y_i)-2D_i-D_{i+1}
	+3[2(y_i-y_{i+1})+D_i+D_{i+1}]}{\alpha_i^2}.
\end{multline}
Grouping the $y$ variables on one side of the equation and the $D$
variables on the other,
\begin{equation}
3\left[\frac{y_{i+2}}{\alpha_{i+1}^2}+y_{i+1}\left(\frac{1}{\alpha_i^2}
	-\frac{1}{\alpha_{i+1}^2}\right)-\frac{y_i}{\alpha_i^2}\right]
	= \frac{D_{i+2}}{\alpha_{i+1}^2}
		+2D_{i+1}\left(\frac{1}{\alpha_i^2}+\frac{1}{\alpha_{i+1}^2}\right)
		+\frac{D_i}{\alpha_i^2}.
\end{equation}
This accounts for the middle rows of Equation~\ref{eq:eqtdi}. The
top and bottom rows come from the initial and final conditions
in Equations~\ref{eq:bslope} and \ref{eq:eslope}. Combining
Equations~\ref{eq:bslope}, \ref{eq:ypp0i}, and \ref{eq:ci},
\begin{align}
2c_1 & = 0\\
3(y_2-y_1) &= 2D_1 + D_2,
\end{align}
which is the first row of Equation~\ref{eq:eqtdi}. Combining
Equations~\ref{eq:eslope}, \ref{eq:yppni}, \ref{eq:ci}, and
\ref{eq:di},
\begin{align}
2c_n+6d_n &= 0\\
3(y_{n+1}-y_n)-2D_n-D_{n+1}+3[2(y_n-y_{n+1})+D_n+D_{n+1}] &= 0\\
3(y_n-y_{n+1}) &= D_n+2D_{n+1},
\end{align}
which is the bottom row of Equation~\ref{eq:eqtd}.

\begin{thebibliography}{9}

\bibitem{mathworld}
Weisstein, Eric W. "Cubic Spline." From
\textit{MathWorld}--A Wolfram Web Resource.
\url{http://mathworld.wolfram.com/CubicSpline.html} (accessed
8/17/2018). Citing: Bartels, R. H.; Beatty, J. C.; and Barsky, B. A.
"Hermite and Cubic Spline Interpolation," Ch. 3 in
\textit{An Introduction to Splines for Use in Computer Graphics
and Geometric Modelling}. San Francisco, CA: Morgan Kaufmann,
pp. 9--17, 1998.

\bibitem{Fritsch}
Fritsch, F. N. and R. E. Carlson. "Monotone Piecewise Cubic Interpolation." SIAM Journal on Numerical Analysis. Vol. 17, 1980, pp.238-–246.

\bibitem{Kahaner}
Kahaner, David, Cleve Moler, Stephen Nash. Numerical Methods and Software. Upper Saddle River, NJ: Prentice Hall, 1988.

\end{thebibliography}

\end{document}
